\documentclass[OPS,lsstdraft,authoryear,toc]{lsstdoc}
\input{meta}

% Package imports go here.

% Local commands go here.

%If you want glossaries
%\input{aglossary.tex}
%\makeglossaries

\title{Charge to the Science Advisory Committee}

% This can write metadata into the PDF.
% Update keywords and author information as necessary.
\hypersetup{
    pdftitle={Charge to the Science Advisory Committee},
    pdfauthor={Michael A.Strauss},
    pdfkeywords={}
}

% Optional subtitle
% \setDocSubtitle{A subtitle}

\input{authors}

\setDocRef{RTN-112}
\setDocUpstreamLocation{\url{https://github.com/lsst/rtn-112}}

\date{\vcsDate}

% Optional: name of the document's curator
% \setDocCurator{The Curator of this Document}

\setDocAbstract{%
The Science Advisory Committee (SAC) is charged with advising the Rubin Observatory Director on matters that bear on the Rubin scientific community or are related to the hardware, system, and survey design.
}

% Change history defined here.
% Order: oldest first.
% Fields: VERSION, DATE, DESCRIPTION, OWNER NAME.
% See LPM-51 for version number policy.
\setDocChangeRecord{%
  \addtohist{1}{2025-12-31}{Released.}{Strauss}
}


\begin{document}

% Create the title page.
\maketitle
% Frequently for a technote we do not want a title page  uncomment this to remove the title page and changelog.
% use \mkshorttitle to remove the extra pages

% ADD CONTENT HERE
% You can also use the \input command to include several content files.

The Science Advisory Committee (SAC) is charged with advising the Rubin Observatory Director on matters that bear on the Rubin scientific community or are related to the hardware, system, and survey design.

\section{Introduction and Background}

The Science Advisory Committee (SAC) has been a standing committee of the Rubin Observatory Project since January 2014, coinciding with the formal start of the federally funded Rubin construction project.
This document was first prepared in late 2022; the current draft reflects updates in late 2025 to reflect the transition to operations.
The SAC advises the Rubin Observatory Director on matters of scientific importance, including those that bear on the Rubin scientific community, or are related to the hardware and system design, the survey planning, policies related to data access, and other aspects of the Rubin Observatory.
The SAC also serves as a mechanism for members of the Rubin scientific community to express concerns and give input to the Rubin leadership.

The SAC will be a standing committee throughout the span of the Legacy Survey of Space and Time (LSST); indeed, as long as the Rubin Observatory is in operation.
The SAC is formally called out in the Rubin Observatory governance as described in the operations plan \citep{RDO-018}.
The SAC is made up of members of the Rubin scientific community who are not otherwise directly funded by the Rubin Observatory.
The Rubin Observatory Director (or their delegate) and the SAC will jointly decide on the questions, tasks, and decision points that the SAC will address.
In the past, the SAC has contributed to the debate on such topics as the Rubin filter design, the decision to use sensors from two vendors in the focal plane, the way in which Rubin leadership communicates with the scientific community, the selection of Community Event Brokers, the review of community white papers on the survey, and many other questions.
It has also helped develop policy and set up committees defining Rubin data rights, the Rubin LSST Science Collaborations, the Survey Cadence Optimization Committee, the Target of Opportunity Advisory Board, and many others.

In operations, the SAC will advise the Director on significant topics that might impact the LSST or the Rubin science community: proposed changes to the survey cadence, for example, or opportunities to effectively address new or emerging science areas not presently captured by the LSST.
The SAC is not an oversight body to the Rubin Observatory.
Rather, the SAC and Observatory will work collaboratively to solve scientific and technical problems, respond to community concerns or ideas, and together foster an enriching/engaging culture in the community of LSST scientists.

\section{Membership}

The SAC shall comprise approximately 12 members, including the Chair.
SAC members should be recognized scientists in their own right, with expertise collectively spanning the full range of scientific areas that Rubin will impact (including at least the “four pillars” identified in the Rubin Science Requirements Document).
Having said that, SAC members do not represent, or formally advocate for, any science collaborations of which they may be members.
Rather, SAC members are expected to bring their own scientific expertise and experience to advocate for the productivity of
the Rubin Observatory overall, and the Rubin scientific community as a whole.

The membership of the SAC should broadly reflect that of the Rubin science community, striving for diverse scientific expertise and a range of backgrounds and experiences (including but not limited to):

\begin{itemize}
\item Science expertise, as mentioned above.
\item Home institution type, including R1 universities, research institutions, and small and/or underserved institutions.
\item Career stage, ranging from postdoctoral researchers to senior faculty.
\item Geography, as elaborated below.
\end{itemize}

As significant Operations partners, at least one member of the SAC shall be from the Chilean scientific community, one member from the UK scientific community, and one member from the French community.

Only those who have LSST data rights are eligible to be SAC members.
SAC members are not required to be a member of any Science Collaborations.
As indicated above, SAC members may not be employed directly by the Rubin Observatory.

The Coordinator of the Rubin Science Collaborations will serve on the SAC ex officio (and will vote like other members when required).
The Rubin Director may bring any number of Rubin staff to SAC meetings as observers.

New SAC members will be selected by the Rubin Director, in consultation with the current SAC members themselves, as well as the relevant authorities in Chile, the UK, and France.
Rubin Observatory will set up a mechanism by which individuals in the Rubin community may indicate their interest to serve, or to nominate others.

SAC members (including the chair) will serve a term of two years; members may be renewed for additional terms.
New members will start July 1 of each year, and no more than half the SAC should rotate off at any given time, to ensure continuity.

Each year, the Rubin Director will select one SAC member to serve as Chair of the SAC, and another SAC member to serve as Vice-Chair.
An individual may serve as Chair and/or Vice-Chair for multiple years.
The SAC Chair will be responsible for calling meetings of the SAC, preparing the minutes and recommendations of the SAC, and helping define the agenda and questions that the SAC will tackle.
The Vice-Chair will assist the Chair in setting the agenda of the meetings, and step in for the Chair if the Chair is unable to attend any given meeting.
The SAC Chair may be invited to serve on other high-level Rubin committees in an ex-officio capacity.

\section{Committee Activities}

The SAC will normally meet four times per year (once per quarter), or more frequently at the request of the Director as called by the SAC Chair.
These meetings will typically be held remotely.
The SAC will also meet once per year at the Rubin Community Workshop (RCW), typically held in July/August.
Rubin will cover full travel costs of SAC members to attend the in-person meeting.
The SAC may ask Rubin personnel and others to give presentations at their meetings.
The SAC session at the RCW will be open to all attendees, but the SAC reserves the right to call an executive session as needed.
Other SAC meetings may be public as well, at the discretion of the SAC Chair.

All in-person meetings of the SAC will be run in a hybrid fashion, allowing participants to take part remotely.

The SAC will produce minutes from its meetings, in the form of a summary of the discussion, highlighting actions and any new concerns or significant items for the observatory and/or the science community.
These minutes will be made public on the SAC website\footnote{\url{https://project.lsst.org/groups/sac/meetings}}.

The time commitment of being a SAC member (other than the Chair) is of order 40 hours (5 working days) per year, but in rare circumstances could be more.

SAC members often serve on other Rubin-related committees or Science Collaboration-related activities; other than the requirement that SAC members not be employed by the Rubin Observatory, there is no formal limit on how many such commitments SAC members can take on.
The SAC can set up standing or ad-hoc subcommittees in consultation and agreement with the Director.
These subcommittees will contain a subset of the SAC and external experts as required.

\section{Relationship to Other Rubin Committees}

Here we list some of the other relevant committees that the SAC’s purview overlaps with, to distinguish their relative responsibilities and interactions.
Note that many of these committees were stood up by the SAC itself.

\begin{itemize}
\item The Survey Cadence Optimization Committee (SCOC; \cite{RTN-089}) works with the team developing the Rubin Operations Simulator and Scheduler to develop a survey plan that maximizes the broad science return of the ten-year LSST. It is a standing committee, and will continue through the life of the LSST. Its recommendations will be reviewed by the SAC (and will be approved by the Rubin leadership). The SAC recommends individuals to serve on the SCOC.
\item The International In-Kind Contribution Evaluation Committee\footnote{\url{https://www.lsst.org/scientists/in-kind-program/contribution-evaluation-committee-cec}} (CEC) reviews proposals from the international community for contributions to Rubin commissioning, operations, or the activities of the Science Collaborations in exchange for data rights. The CEC will continue to be a standing committee throughout LSST operations, having a minor advisory role in the annual evaluation process of international contributions.
\item The User’s Committee \citep{RDO-051} works with the Community Science Team\footnote{\url{https://rubinobservatory.org/for-scientists/committees-teams/community-science-team}} to give a science community perspective of the Rubin Science Platform. It is a channel for the community to suggest improvements to the RSP and the pipelines that populate it with data. The User’s Committee is a standing committee, and will continue throughout LSST operations.
\item The Target of Opportunity (ToO) Advisory Board \citep{RTN-110} advises Rubin leadership on various aspects of the ToO process during Rubin Operations.
\item The Rubin Science Collaborations\footnote{\url{https://rubinobservatory.org/for-scientists/science-collaborations}} are semi-autonomous, consisting of one hundred to over one thousand members each, which are preparing themselves to carry out science analyses with the LSST data. The Science Collaborations are organized following guidelines described in the Science Collaboration Federation Charter\footnote{\url{https://ls.st/sc-fed-charter-docushare}}. At this writing, there are eight Science Collaborations. Representatives of each Science Collaboration serve on a Science Collaboration Council, which is chaired by the Science Collaboration Coordinator. The Science Collaboration Coordinator serves as an ex-officio member of the SAC.
\end{itemize}

\section{Updating this charge}

This document should be reviewed by the SAC and the Director once per year.
Any changes will be implemented upon a two-thirds vote of the SAC, and approval from the Director.


\appendix

\section{Acknowledgements}

This material is based upon work supported in part by the National Science Foundation through Cooperative Agreements AST-1258333 and AST-2241526 and Cooperative Support Agreements AST-1202910 and AST-2211468 managed by the Association of Universities for Research in Astronomy (AURA), and the Department of Energy under Contract No.\ DE-AC02-76SF00515 with the SLAC National Accelerator Laboratory managed by Stanford University.
Additional Rubin Observatory funding comes from private donations, grants to universities, and in-kind support from LSST-DA Institutional Members.

% Include all the relevant bib files.
% https://lsst-texmf.lsst.io/lsstdoc.html#bibliographies
\section{References} \label{sec:bib}
\renewcommand{\refname}{} % Suppress default Bibliography section
\bibliography{local,lsst,lsst-dm,refs_ads,refs,books}

% Make sure lsst-texmf/bin/generateAcronyms.py is in your path
\section{Acronyms} \label{sec:acronyms}
\input{acronyms.tex}
% If you want glossary uncomment below -- comment out the two lines above
%\printglossaries





\end{document}
